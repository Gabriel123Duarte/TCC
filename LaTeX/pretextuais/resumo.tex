\begin{resumo}
Programação dinâmica é um dos principais temas de problemas das competições de programação, e seu uso é bem discutido na literatura. Porém, as técnicas de otimização são pouco discutidas na literatura e estas estão em alta nas últimas competições de programação que acontecem em todo o mundo. O objetivo deste trabalho é a criação de um material didático que explique algumas das principais otimizações de programação dinâmica, abordando suas características e particularidades, com o intuito de tornar o leitor capaz de aplicar esses conceitos em diversos problemas que compartilham das propriedades similares dos problemas que aqui serão demonstrados. Inicialmente uma fundamentação teórica é feita com a finalidade de esclarecer todos os termos que são importantes no decorrer do trabalho. Além desta parte técnica dos assuntos, foi feito um levantado teórico sobre o ensino de algoritmos e programação, a fim de deixar as explicações das técnicas mais formais, minimizando a dificuldade de entendimento. Para apresentar cada técnica, foi desenvolvida uma metodologia que permite ao leitor acompanhar toda a linha de raciocínio, desde a descrição de um problema proposto, até a sua solução final, após a inclusão da otimização. Com isso, a leitura fica mais simplificada, pois como otimização de programação dinâmica é considerado um tema complexo e não recomendado para iniciantes em programação, houve a preocupação de criar um texto capaz de guiar os pensamentos do leitor e que tenha a capacidade de generalizar cada técnica para outros problemas. Dentre as otimizações possíveis, foram escolhidas as mais comuns nas maratonas de programação. São elas: Redução de Espaço, Estrutura de Dados RMQ, \textit{Divide and Conquer Optimization}, \textit{Knuth Optimization} e \textit{Convex Hull Trick}.
\end{resumo}	
