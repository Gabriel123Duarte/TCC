%%%%%%%%%%%%%%%%%%%%%%%%%%%%%%%%%%%%%%%%%%%%%%%%%%%%%%%%%%%%%%%%%%%%%%%%%%%%%%%
% Este arquivo faz parte do template de Relatório Parcial baseado nas normas da ABNT 
%  voltado para alunos da UEFS
% Desenvolvimento: Danilo de Oliveira Gonçalves
% Adaptação final: João Carlos Nunes Bittencourt
% Data: 31/03/2011
% Atualização: 30/11/2011
% Descrição do arquivo:
%   Esse arquivo apresenta as definições de constantes que formarão a capa e 
%   a folha de rosto. Siga as instruções e modifique de acordo com o que
%   lhe foi orientado.
%%%%%%%%%%%%%%%%%%%%%%%%%%%%%%%%%%%%%%%%%%%%%%%%%%%%%%%%%%%%%%%%%%%%%%%%%%%%%%%

% ---------- Preambulo ----------
\instituicao{Centro Universitário Serra dos Órgãos – UNIFESO} % nome da instituicao
\fundacao {Fundação Educacional Serra dos Órgãos – FESO} % Fundação
\departamento{CENTRO DE CIÊNCIAS E TECNOLOGIA – CCT}
\graduacao{Curso DE BACHARELADO EM CIÊNCIA DA COMPUTAÇÃO} % nome do curso
\curso{Ciência da Computação}

\documento{Trabalho de Conclusão de Curso} % tipo de documento
\titulacao{Bacharel} % [Bacharel]

\titulo{Técnicas de otimização de programação dinâmica} % titulo do trabalho em portugues
%\subtitulo{Sub-título, se necessário} % caso necessário um sub-título, utilize este campo
\title{Title in English} % titulo do trabalho em ingles

\autor{Gabriel Lagoa Duarte} % autor do trabalho
\cita{DUARTE, Gabriel} % sobrenome (maiusculas), nome do autor do trabalho

\palavraschave{Programação dinâmica. Otimização. Maratona de programação. \textit{Convex Hull Trick}.} % palavras-chave do trabalho, separados por ponto
\keywords{Dynamic Programming. Optimization. Programming contest. Convex Hull Trick.} % palavras-chave do trabalho em ingles, separados por ponto

\comentario{\UEFSdocumentodata\ apresentado ao \ABNTinstituicaodata\ como requisito obrigatório para obtenção do título de \UEFStitulacaodata\ em \UEFScursodata.}

\orientador{Rafael Gomes Monteiro} % nome do orientador do trabalho
%\orientador[Orientadora:]{Nome da Orientadora} % <- no caso de orientadora, usar esta sintaxe
\coorientador{} % nome do co-orientador do trabalho, caso exista
%\coorientador[Co-orientadora:]{Nome da Co-orientadora} % <- no caso de co-orientadora, usar esta sintaxe

\local{Teresópolis} % cidade
\data{2017} % ano

% Termo Aprovação

\textoaprovacao{\UEFSdocumentodata\ aprovado como requisito parcial para obtenção do título de \UEFStitulacaodata\ no \ABNTinstituicaodata\ pela banca examinadora:}

\primeiroassina{Rafael Gomes Monteiro (Orientador), M.Sc.}

\segundoassina{Eugênio da Silva, D.Sc.}

\terceiroassina{Hermano Lourenço Souza Lustosa, M.Sc.}

\localdia{Teresópolis}
\mesdia{22 de novembro de 2017}

% DECLARAÇÃO DE PRÓPRIA AUTORIA
\tituloautoria{DECLARAÇÃO DE PRÓPRIA AUTORIA}
\dataautoria{Teresópolis, XX/XX/2017}
\textoautoria{Eu, \ABNTautordata, declaro para fins de conclusão do Curso de Bacharelado em Ciência da Computação do UNIFESO, que este \UEFSdocumentodata\ é de minha própria autoria, estando ciente das consequências disciplinares a que estarei sujeito caso seja comprovada fraude ou má-fé. Sem mais, subscrevo-me,}
