\begin{abstract}
Dynamic programming is one of the main themes of programming contests, and its use is well discussed in the literature. However, optimization techniques are scarcely discussed in the literature and these are on the rise in the latest programming contests that take place around the world. The objective of this work is the creation of a didactic material that explains some of the main optimizations of dynamic programming, approaching its characteristics and particularities, with the intention of making the reader able to apply these concepts in several problems that share the similar properties of the problems that will be demonstrated here. Initially a theoretical referencial is made with the purpose of clarifying all the terms that are important in the course of the work. Besides this technical part of the subjects, it was made a search in  the literature about on the teaching of algorithms and programming, in order to leave the explanations of the techniques more formal, minimizing the difficulty of understanding. To present each technique, a methodology was developed that allows the reader to follow the whole line of reasoning, from the description of a proposed problem, to its final solution, after including the optimization. With this, the reading becomes more simplified, because as dynamic programming optimization is considered a complex theme and not recommended for beginners
in programming, there was the concern to create a text capable of guiding the thoughts of the reader and make he to has the ability to generalize each technique for other problems. Among all optimizations possibilities, the most common in programming contests were chosen. Are they: Space reduction, RMQ data structures, Divide and Conquer Optimization, Knuth Optimization, Convex Hull Trick.
\end{abstract}
