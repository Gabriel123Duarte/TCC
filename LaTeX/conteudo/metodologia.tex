
\chapter{Metodologia}
\label{chap:metodo}
Após a análise dos trabalhos levantados no Seção \ref{sec:ensino}, pode-se destacar os trabalhos elaborados por Szlávi e Zsakó (2003) e Vihavainen, Paksula e Luukkainen (2011). No primeiro, um dos métodos discutidos aborda uma forma mais metodológica para transmitir uma informação, onde em cada etapa do ensino há uma clara divisão. No segundo trabalho é apresentada uma forma de ensino baseada no desenvolvimento de uma linha de raciocínio, em que o aluno ou o leitor vai acompanhando todos os passos que o instrutor fez até chegar em uma conclusão.

A metodologia aplicada no próximo capítulo tomará como base esses dois trabalhos, com uma organização baseada em tópicos, porém sempre fazendo o leitor entender toda a linha de raciocínio. Serão inseridas algumas pausas a fim de deixar quem está lendo pensar em uma solução para um problema, antes de ser apresentada a resposta. Além disso, alguns exercícios serão propostos para quem desejar praticar o que está sendo demonstrado.

A estrutura de cada seção será dividida em seis partes, sendo elas:
\vspace{-0.1in}
\begin{itemize}[leftmargin=-.001in]
\item \textbf{Problema:} Será apresentado um problema com todas as informações que são importantes para que o mesmo seja resolvido através de programação dinâmica;
\vspace{-0.1in}
\item \textbf{Solução ingênua:}  A partir da descrição do problema será elaborada uma solução que utilize programação dinâmica, juntamente com uma análise de complexidade, podendo ser temporal e/ou de espaço;
\vspace{-0.1in}
\item \textbf{Análise de particularidades:} Após uma solução ter sido desenvolvida serão levantadas algumas características do problema, principalmente com relação à forma que os cálculos estão sendo desenvolvidos, e como estão sendo armazenadas as informações, para que, com estes dados, possa ser feita alguma otimização. Neste ponto o leitor já teria condições de começar a pensar no que pode ser feito para melhorar a solução;
\vspace{-0.1in}
\item \textbf{“Nome da técnica”:} Com todas as análises feitas, será apresentada a técnica em questão, mostrando o seu funcionamento e como esta pode ser aplicada no problema. O nome da seção irá variar de acordo com cada técnica;
\vspace{-0.1in}
\item \textbf{Benefícios:} Tendo aplicada a otimização, uma análise dos benefícios será feita juntamente com a apresentação e discussão das novas complexidades;
\vspace{-0.1in}
\item \textbf{Código final:} Ao final, uma forma de implementação do problema com a linguagem C++ será apresentada, com o objetivo de sanar possíveis dúvidas do leitor.
\end{itemize}

\vspace{-0.1in}
Entre cada tópico podem existir alguns momentos de reflexão com sugestões de tarefas que o leitor pode fazer para tentar exercitar os conteúdos. Estas serão apresentadas conforme o seguinte modelo:

\vspace{0.1in}
\tikz[baseline=-4pt,align=left]\node[draw,minimum width=12.5cm,minimum height=4ex] {\textit{Sugestão de tarefa ao leitor...}};