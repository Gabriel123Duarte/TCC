
\chapter{Metodologia}
\label{chap:metodo}

Após a análise dos trabalhos levantados no seção \ref{sec:ensino}, pode-se destacar os trabalhos elaborados por Szlávi e Zsakó (2003) e Vihavainen, Paksula e Luukkainen (2011). O primeiro em um dos métodos discutidos é abordado uma forma mais metodológica na hora de transmitir uma informação, onde em cada etapa do ensino há uma clara divisão. No segundo trabalho é mostrado uma maneira de ensino baseada no desenvolvimento de uma linha de raciocínio, em que o aluno ou o leitor, vai acompanhando todos os passos que o instrutor fez até chegar numa conclusão.

A metodologia aplicada no próximo capítulo tomará como base esses dois trabalhos, levando uma linha bem divida entre os tópicos, porém sempre fazendo o leitor entender toda a linha de raciocínio, com algumas pausas a fim de deixar quem está lendo pensar em uma solução para um problema, antes de ser mostrado a resposta, além disso alguns exercícios serão propostos para quem desejar praticar o que está sendo demonstrado.

A estrutura de cada seção será dividida em seis partes, sendo elas:

\begin{itemize}
\item \textbf{Problema:} Será apresentado um problema com todas as informações que são importantes para ser resolvido através de programação dinâmica;
\item \textbf{Solução ingênua:}  A partir da descrição do problema será elaborada uma solução que utilize programação dinâmica , juntamente com a análise de complexidade, podendo ser temporal e/ou de espaço;
\item \textbf{Análise de particularidades:} Após uma solução ter sido desenvolvida será levantado as características do problema, sendo elas a forma que os cálculos estão sendo desenvolvidos, como está sendo armazenado as informações, para que com estes dados possa ser feita alguma otimização. Neste ponto o leitor já começa a imaginar o que pode ser realizado para melhorar a solução;
\item \textbf{“Nome da técnica”:} Com todas as análises feitas, será formulado a técnica em questão, mostrando o seu funcionamento e como esta pode ser aplicada no problema;
\item \textbf{Benefícios:} Tendo aplicado a otimização, uma análise dos benefícios será feita juntamente com as novas complexidades;
\item \textbf{Código final:} Ao final uma forma de implementação do problema com a linguagem C++ será apresentada, com o objetivo de sanar possíveis dúvidas do leitor.
\end{itemize}

Entre cada tópico pode existir algumas paradas com sugestões de tarefas que o leitor pode fazer, para tentar exercitar os conteúdos, estas serão apresentadas conforme o seguinte modelo:

\tikz[baseline=-4pt,align=left]\node[draw,minimum width=12.5cm,minimum height=4ex] {\textit{Sugestão de tarefa ao leitor...}};