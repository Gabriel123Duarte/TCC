
\chapter{Histórico e trabalhos relacionados}
\label{chap:historico}

\textcolor{red}{Capítulo destino a fazer o levanto sobre programação dinâmica, desde seu surgimento
até os dias atuais, mostrando os trabalhos na área e onde tem sido utilizada.}

Citar os trabalhos, dizendo o que cada um fez.

O "porém": nada focado nas otimizações X, Y e Z
- onde é usado além das maratonas


\section{Trabalhos Relacionados}
Nesta seção são apresentados alguns trabalhos que tem um objetivo similar ao deste projeto. Em todos os encontrados a intenção do autor foi criar um conteúdo teórico sobre algoritmos e programação. Apesar de apenas um trabalho possuir conteúdo relacionado a programação dinâmica, todos eles são úteis para a elaboração da metologia a ser desenvolvida.

Tommasini (201-) em seu trabalho visa ensinar programação dinâmica para quem está iniciando nesta área. Sua maior motivação foi a falta de um bom material didático sobre esse tema. Seu trabalho tem um foco muito didático, apresentando diversas técnicas, problemas com soluções e propondo vários exercícios para o leitor praticar o que foi ensinado. Ao final do trabalho é apresentado uma lista de problemas de maratonas de programação sobre os assuntos desenvolvidos no texto.


No trabalho proposto por Dalalio (2013), foi realizado um estudo de algoritmos e estruturas de dados para a resolução de problemas relacionados a \textit{String Matching}\footnote{https://en.wikipedia.org/wiki/String\_searching\_algorithm}. Seu texto é bem didático, mostrando diversos problemas e as diversas formas de resolução, apresentando a complexidade e seu código. Ao final diversos problemas são propostos para que o leitor os resolva utilizando os conceitos elaborados no texto.


Couto (2016) desenvolveu um trabalho voltado ao ensino de algoritmos, seu foco principal são os algoritmos aplicados em sequências de caracteres. Seu texto ficou dividido entre diversos capítulos, onde cada um deles apresentava uma estrutura de dados diferente, assim podendo mostrar ao leitor quando é melhor utilizar uma em relação as outras.






