
\chapter{Metodologia}
\label{chap:metodo}
\textcolor{red}{Explicar a linguagem que será utilizada (tanto do texto corrido, quanto a linguagem de programação, pseudocódigo, etc)}

Após a análise dos trabalhos levantados no seção \ref{sec:ensino}, pode-se destacar os trabalhos elaborados por Szlávi e Zsakó (2003) e Vihavainen, Paksula e Luukkainen (2011), onde o primeiro em um dos métodos discutidos é abordado uma forma mais metodológica de ensino, onde em cada etapa do ensino há uma clara divisão. No segundo trabalho é mostrado uma maneira de ensino baseado no desenvolvimento de uma linha de raciocínio, onde o aluno ou o leitor, vai acompanhando todos os passos que o instrutor fez até chegar numa conclusão.

A metodologia aplicada no próximo capítulo, tomará como base esses dois trabalhos, levando uma linha bem divida entre os tópicos, porém sempre fazendo o leitor entender toda a linha de raciocínio, com algumas pausas afim de deixar quem está lendo pensar em uma solução para um problema, antes de ser mostrado a resposta, além disso alguns exercícios serão propostos para quem desejar praticar o que está sendo demostrado.

A estruturação será feita da seguinte maneira:


Inicialmente sera ab

- Apresentação de um problema

- Formulação da solução utilizando DP

- Análise das particularidades

- Inserir a técnica

- Análise dos benefícios

- Código final otimizado

