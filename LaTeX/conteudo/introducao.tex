\chapter{Introdu\c{c}\~ao}
\label{chap:introducao}

As maratonas de programação são competições que exigem criatividade, trabalho em equipe e a capacidade de resolver problemas sob pressão \cite{Piekarski2015}. Essas competições possuem diversos problemas, que normalmente simulam uma situação real, onde um profissional da área deveria estar apto a resolver, se este aparecesse em um ambiente de trabalho, por exemplo. Cada exercício que deve ser resolvido faz parte de alguma subárea de conhecimento que está inclusa no grande universo que é a Ciência da Computação. Dentre todos os temas possíveis para os problemas, destacam-se os seguintes: grafos, estruturas de dados, geometria computacional e programação dinâmica. Estes ganham um foco maior pelos estudantes e autores das provas, devido ao fato de que todos eles são conteúdos amplos e comumente utilizados fora das competições em problemas reais. Sendo assim, os problemas conseguem ficar cada vez mais concretos.

Dentre as técnicas mais comuns nas provas de maratona de programação, deve-se dar um destaque maior à programação dinâmica, pois ela está presente em todas as competições mais importantes que acontecem ao redor do mundo. Além disso, esta técnica é muito utilizada na indústria, quando há a necessidade de otimizar algum código, pois muitas vezes é possível realizar isso com a utilização de programação dinâmica. Por esses motivos, programação dinâmica é um tema bem conhecido e discutido, estando presente na grade curricular de muitos cursos e em diversos livros, principalmente nos de projeto e análise de algoritmos.

Apesar da programação dinâmica ser uma técnica que consegue otimizar várias soluções, somente o seu uso nem sempre é suficiente. Às vezes ainda é preciso algo mais eficiente, como pode ser visto no problema \textit{Internet Trouble}\footnote{http://maratona.ime.usp.br/hist/2016/index.html}, que fez parte da XXI Maratona de Programação que é promovida pela\sigla{SBC}{Sociedade Brasileira de Computação}(Sociedade Brasileira de Computação) e no problema \textit{Branch Assignment}\footnote{https://icpc.baylor.edu/worldfinals/problems/icpc2016.pdf}, que fez parte da final mundial, realizada pelo\sigla{ICPC}{International Collegiate Programming Contest}(do inglês, \textit{International Collegiate Programming Contest}). Nos dois problemas citados, a solução esperada, segundo os próprios autores, seria através de programação dinâmica. Porém, utilizar a solução mais intuitiva consumiria muito tempo de processamento para encontrar a resposta da questão, fazendo com que esta não seja uma solução viável. Se for feita uma análise mais profunda dos problemas, é possível perceber que estes possuem algumas particularidades que possibilitam realizar uma otimização da solução que utiliza programação dinâmica, e esta seria suficiente para resolver o problema.

As técnicas de otimização de programação dinâmica são normalmente assuntos mais avançados, que requerem um pouco mais de maturidade em programação e de algumas estruturas de dados e algoritmos. Sendo assim, esses temas não são amplamente explorados na literatura e, como algumas técnicas são derivadas por pessoas que precisavam otimizar algum problema específico que estava sendo resolvido, muitas vezes essas não são formalizadas para que futuramente outros possam utilizar a mesma solução encontrada, o que torna o aprendizado dessas técnicas mais difícil. Até a data da confecção deste trabalho, não foi encontrado na literatura nenhum livro que aborde esse assunto. Algumas publicações na internet podem ser encontradas, porém, não há um material completo que explique todas as características e formas de utilização.

O objetivo deste trabalho é a criação de um material didático que formalize e explique algumas das principais otimizações de programação dinâmica, abordando suas características e particularidades, com o intuito de tornar o leitor capaz de aplicar esses conceitos em diversos cenários que compartilham das propriedades similares dos problemas que aqui serão demostrados. Apesar deste trabalho ter um viés teórico com uma proposta mais didática, este não é recomendado para iniciantes em programação. As técnicas aqui elucidadas requerem que o leitor possua um conhecimento amplo em programação dinâmica, análise de algoritmos, estruturas de dados e matemática, para poder compreender todo o texto. Mesmo sendo explicado todos os passos das otimizações, os assuntos que são considerados iniciantes ou intermediários não serão abordados.

Este trabalho está organizado da seguinte forma: no Capítulo \ref{chap:fundamentacao} são discutidos os principais assuntos necessários para o melhor compreendimento das técnicas, que são: programação dinâmica, complexidade de algoritmos e otimização. Além disso, nesse capítulo são apresentados alguns trabalhos que visam formalizar algumas maneiras de como ensinar algoritmos e programação. No Capítulo \ref{chap:historico} são apresentados alguns trabalhos de conclusão de curso que possuem um objetivo similar ao deste, mesmo que não em programação dinâmica, mas sim no ensino de algoritmos. No Capítulo \ref{chap:metodo} é definida a metodologia de ensino aplicada no trabalho. O Capítulo \ref{chap:desenvolvimento} é responsável pela apresentação e explicação de todas as otimizações selecionadas. Por fim, são feitas as considerações finais no Capítulo \ref{chap:conclusao}.

