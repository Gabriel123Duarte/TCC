\chapter{Considerações Finais}
\label{chap:conclusao}

Através da metodologia proposta, esse trabalho trouxe técnicas avançadas de programação dinâmica. É esperado que este tenha servido como inspiração e como uma base para quem deseja se aprofundar nessa grande área. O objetivo principal era a construção de um material didático completo onde um leitor que não possua um conhecimento sobre otimizações, consiga apenas com este material entender os temas propostos e aplicar os conceitos em diversos problemas.

Apesar do trabalho ter um conteúdo mais complexo, com assuntos que normalmente apenas pessoas mais experientes conseguem dissolver, acredita-se que o objetivo principal do trabalho foi concluído, tornando esses conteúdos mais acessíveis, pois por mais que as técnicas propostas sejam densas, no decorrer de todo o texto, as explicações foram feitas da forma mais simplificada e detalhada possível, com intuito de sanar todas as possíveis dúvidas que o leitor possa ter.

Espera-se que este projeto seja relevante e útil para todos que venham a ler, principalmente as competidores de maratonas de programação, onde esses temas são comumente abordados.