\documentclass{abnt-uefs} % Classe de formatação da UEFS
\usepackage[utf8]{inputenc}
%\usepackage[T1]{fontenc}
\usepackage[call=authordata,alf,abnt-emphasize=bf,abnt-etal-text=emph,abnt-and-type=&,abnt-etal-list=3,abnt-etal-cite=3,recuo=0.0cm]{abntcite}
\usepackage[brazil]{babel}
\usepackage{mathptmx}
\usepackage[pdftex]{graphicx}
%\usepackage[small,bf]{caption}
\usepackage{longtable}
\usepackage{array}
\usepackage{amssymb,amsmath,amsthm,amsfonts}
%\usepackage{textcomp}
%\usepackage{textcase} 
\usepackage{float} % para as figuras ficarem onde foram colocadas no latex - deve colcoar na figura [H]
\usepackage{color}
%\usepackage{url}
\usepackage{nameref}
\usepackage{color}   %May be necessary if you want to color links
\usepackage{amsmath}

%Gambiarra das imagens
\usepackage{placeins}
\usepackage{tikz}
\let\Oldsection\section
\renewcommand{\section}{\FloatBarrier\Oldsection}

\let\Oldsubsection\subsection
\renewcommand{\subsection}{\FloatBarrier\Oldsubsection}

\let\Oldsubsubsection\subsubsection
\renewcommand{\subsubsection}{\FloatBarrier\Oldsubsubsection}

%Tirar negrito do título das imagens


% Colocar links nos capítulos
%\usepackage{hyperref}
%\hypersetup{
%	colorlinks=true, %set true if you want colored links
%	linktoc=all,     %set to all if you want both sections and subsections linked
%	linkcolor=blue,  %choose some color if you want links to stand out
%}
%\makeatletter
%\newcommand*{\updatelabelname}[1]{% For correcting \@currentlabelname, if needed
%	\xdef\@currentlabelname{#1}}
%\makeatother


%%%%%%%%%%%%%%%%%%%%%%%%%%%%%%%%%%%
% Configuração de código fonte
% Colorido
\usepackage{color}
\usepackage{bold-extra}
\usepackage{listings}
\definecolor{dkgreen}{rgb}{0,0.6,0}
\definecolor{gray}{rgb}{0.5,0.5,0.5}
\definecolor{mauve}{rgb}{0.58,0,0.82}
\lstset{
	language=c++, % Replace
	basicstyle={\footnotesize\ttfamily},
	keywordstyle={\bfseries\color{blue}},
	commentstyle=\color{dkgreen},
	stringstyle={\slshape\color{mauve}},
	numberstyle=\footnotesize,
	numbers=left,
	showstringspaces=false,
	breaklines=true,
	tabsize=4,
	frame=tb
}

\renewcommand{\lstlistingname}{Algoritmo}% Listing -> Algorithm
\renewcommand{\lstlistlistingname}{Lista de \lstlistingname s}
\makeatletter
\renewcommand{\l@lstlisting}[2]{%
	%\@dottedtocline{1}{0em}{1.5em}{\lstlistingname\ #1}{#2}%
	\leftskip 3.1em
	\rightskip 1.6em
	\parfillskip -\rightskip
	\parindent 0em
	\@tempdima 2.0em
	\advance\leftskip \@tempdima \nobreak\hskip -\leftskip %%Tinha \null\nobreak\..
	{Algoritmo \normalfont #1}\nobreak \tabfillnum{#2}
	
}
\makeatother


			
% Preto e branco
%\usepackage{xcolor}
%\usepackage{bold-extra}
%
%\usepackage{listings}
%\lstset{
%	language=c, %Replace
%	basicstyle={\footnotesize\ttfamily},
%	keywordstyle=\bfseries,
%	commentstyle=\color{black!75},
%	stringstyle=\slshape,
%	numberstyle=\footnotesize,
%	numbers=left,
%	showstringspaces=false,
%	breaklines=true,
%	tabsize=4,
%	frame=tb
%}



\graphicspath{{./figuras/}} % Diretório padrão de figuras.
%%%%%%%%%%%%%%%%%%%%%%%%%%%%%%%%%%%%%%%%%%%%%%%%%%%%%%%%%%%%%%%%%%%%%%%%%%%%%%%
% Este arquivo faz parte do template de Relatório Parcial baseado nas normas da ABNT 
%  voltado para alunos da UEFS
% Desenvolvimento: Danilo de Oliveira Gonçalves
% Adaptação final: João Carlos Nunes Bittencourt
% Data: 31/03/2011
% Atualização: 30/11/2011
% Descrição do arquivo:
%   Esse arquivo apresenta as definições de constantes que formarão a capa e 
%   a folha de rosto. Siga as instruções e modifique de acordo com o que
%   lhe foi orientado.
%%%%%%%%%%%%%%%%%%%%%%%%%%%%%%%%%%%%%%%%%%%%%%%%%%%%%%%%%%%%%%%%%%%%%%%%%%%%%%%

% ---------- Preambulo ----------
\instituicao{Centro Universitário Serra dos Órgãos – UNIFESO} % nome da instituicao
\fundacao {Fundação Educacional Serra dos Órgãos – FESO} % Fundação
\departamento{CENTRO DE CIÊNCIAS E TECNOLOGIA – CCT}
\graduacao{Curso DE BACHARELADO EM CIÊNCIA DA COMPUTAÇÃO} % nome do curso
\curso{Ciência da Computação}

\documento{Trabalho de Conclusão de Curso} % tipo de documento
\titulacao{Bacharel} % [Bacharel]

\titulo{Técnicas de otimização de programação dinâmica} % titulo do trabalho em portugues
%\subtitulo{Sub-título, se necessário} % caso necessário um sub-título, utilize este campo
\title{Title in English} % titulo do trabalho em ingles

\autor{Gabriel Lagoa Duarte} % autor do trabalho
\cita{DUARTE, Gabriel} % sobrenome (maiusculas), nome do autor do trabalho

\palavraschave{Programação dinâmica. Otimização. Maratona de programação. \textit{Convex Hull Trick}.} % palavras-chave do trabalho, separados por ponto
\keywords{Dynamic Programming. Optimization. Programming contest. Convex Hull Trick.} % palavras-chave do trabalho em ingles, separados por ponto

\comentario{\UEFSdocumentodata\ apresentado ao \ABNTinstituicaodata\ como requisito obrigatório para obtenção do título de \UEFStitulacaodata\ em \UEFScursodata.}

\orientador{Rafael Gomes Monteiro} % nome do orientador do trabalho
%\orientador[Orientadora:]{Nome da Orientadora} % <- no caso de orientadora, usar esta sintaxe
\coorientador{} % nome do co-orientador do trabalho, caso exista
%\coorientador[Co-orientadora:]{Nome da Co-orientadora} % <- no caso de co-orientadora, usar esta sintaxe

\local{Teresópolis} % cidade
\data{2017} % ano

% Termo Aprovação

\textoaprovacao{\UEFSdocumentodata\ aprovado como requisito parcial para obtenção do título de \UEFStitulacaodata\ no \ABNTinstituicaodata\ pela banca examinadora:}

\primeiroassina{Rafael Gomes Monteiro (Orientador), M.Sc.}

\segundoassina{Eugênio da Silva, D.Sc.}

\terceiroassina{Hermano Lourenço Souza Lustosa, M.Sc.}

\localdia{Teresópolis}
\mesdia{22 de novembro de 2017}

% DECLARAÇÃO DE PRÓPRIA AUTORIA
\tituloautoria{DECLARAÇÃO DE PRÓPRIA AUTORIA}
\dataautoria{Teresópolis, XX/XX/2017}
\textoautoria{Eu, \ABNTautordata, declaro para fins de conclusão do Curso de Bacharelado em Ciência da Computação do UNIFESO, que este \UEFSdocumentodata\ é de minha própria autoria, estando ciente das consequências disciplinares a que estarei sujeito caso seja comprovada fraude ou má-fé. Sem mais, subscrevo-me,}
 % Elementos da capa
\begin{document}	
	% Numeração de algoritmos e equações
	\renewcommand{\thelstlisting}{\arabic{lstlisting}}
	\renewcommand{\theequation}{\arabic{equation}}		
	
	
    \pagestyle{empty}
    \DeclareGraphicsExtensions{.jpg,.pdf,.mps,.png,.bmp,.eps}
    \capa % geração automática da capa
    \folhaderosto % geração automática da folha de rosto
    \termodeaprovacao % geração automática do termo de aprovação
    %%%%%%%%%%%%%%%%%%%%%%%%%%%%%%%%%%%%%%%%%%%%%%%%%%%%%%%%%%%%%%%%%%%%%%%%%%%%%%%
%% Este arquivo faz parte do template de TCC baseado nas normas da ABNT 
%%  voltado para alunos da UEFS
%% Desenvolvimento: Danilo de Oliveira Gonçalves
%% Adaptação final: João Carlos Nunes Bittencourt
%% Data: 31/03/2011
%%%%%%%%%%%%%%%%%%%%%%%%%%%%%%%%%%%%%%%%%%%%%%%%%%%%%%%%%%%%%%%%%%%%%%%%%%%%%%%


% dedicatória (opcional)
\begin{dedicatoria}
\hfill \textit{Dedico esta monografia a minha família,\\pelo apoio fornecido e aos meus amigos das maratonas de programação.\\}
\end{dedicatoria}

%\vfill

%\begin{flushright}
%\hfill \textit{Dedico esta monografia a minha família,\\pelo apoio fornecido e aos meus amigos.\\}
%\end{flushright}

%\vspace*{1cm}

%\clearpage 

    % agradecimentos (opcional)
\begin{agradecimentos}
A Deus por minha vida, família e amigos.

Ao meu orientador Rafael Monteiro, pelo suporte e suas correções neste trabalho. Além disso, pelo incentivo e apoio nas maratonas de programação.

Aos meus pais, pelo amor, incentivo e todo o apoio concedido durante toda a minha vida.

E a todos que direta ou indiretamente fizeram parte da minha formação, o meu muito obrigado.
\end{agradecimentos}

    
% epigrafe (opcional)
\begin{epigrafe}
\textit{"The mind is the limit. As long as the mind can envision the fact that you can do something, you can do it, as long as you really believe 100 percent."}

\hfill{\textit{(Arnold Schwarzenegger)}}
\end{epigrafe}

%\vfill

%\begin{flushright}
%\hfill \textit{Dedico esta monografia a minha família,\\pelo apoio fornecido e aos meus amigos.\\}
%\end{flushright}

%\vspace*{1cm}

%\clearpage 

    \listadefiguras % geracao automatica da lista de figuras
    \listadetabelas % geracao automatica da lista de tabelas
    \listadesimbolos % geracao automatica da lista de símbolos
    \listadesiglas % % geracao automatica da lista de siglas
    \lstlistoflistings  % % geracao automatica da lista de algoritmos

    \begin{resumo}
Programação dinâmica é um dos principais temas de problemas das competições de programação, seu uso é bem discutido na literatura. Porém, as técnicas de otimizações, são pouco discutidas na literatura e estas estão em alta nas últimas competições de programação que acontecem em todo o mundo. O objetivo deste trabalho é a criação de um material didático que explique algumas das principais otimizações de programação dinâmica, abordando suas características e particulares, com o intuito de tornar o leitor capaz de aplicar esses conceitos em diversos problemas que compartilham das propriedades similares dos problemas que aqui serão demostrados. Inicialmente uma fundamentação teórica é feita com a finalidade de esclarecer todos os termos que são importantes no decorrer do trabalho, além desta parte técnica dos assuntos, foi feito um levantado teórico sobre o ensino de algoritmos e programação, a fim de deixar as explicações das técnicas mais formais, porém sem ser algo difícil para o leitor. Para apresentar cada técnica, foi desenvolvida uma metodologia que tenta deixar o leitor acompanhar toda a linha de raciocínio, desde a descrição de um problema proposto, até a sua solução final, após a inclusão da otimização. Com isso, a leitura fica mais simplificada, pois como otimização de programação dinâmica é considerado um tema complexo e não recomendado para iniciantes em programação, houve a preocupação de criar o texto capaz de guiar os pensamentos do leitor e que tenha a capacidade de generalizar cada técnicas para outros problemas. Dentre as otimizações possíveis, foram escolhidas as mais comuns nas maratonas de programação, são elas: Redução de Espaço, Estrutura de Dados RMQ, \textit{Divide and Conquer Optimization}, \textit{Knuth Optimization} e \textit{Convex Hull Trick}.
\end{resumo}	

    \begin{abstract}
Dynamic programming is one of the main themes of programming contests, and its use is well discussed in the literature. However, optimization techniques are scarcely discussed in the literature and these are on the rise in the latest programming contests that take place around the world. The objective of this work is the creation of a didactic material that explains some of the main optimizations of dynamic programming, approaching its characteristics and particularities, with the intention of making the reader able to apply these concepts in several problems that share the similar properties of the problems that will be demonstrated here. Initially a theoretical referencial is made with the purpose of clarifying all the terms that are important in the course of the work. Besides this technical part of the subjects, it was made a search in  the literature about on the teaching of algorithms and programming, in order to leave the explanations of the techniques more formal, minimizing the difficulty of understanding. To present each technique, a methodology was developed that allows the reader to follow the whole line of reasoning, from the description of a proposed problem, to its final solution, after including the optimization. With this, the reading becomes more simplified, because as dynamic programming optimization is considered a complex theme and not recommended for beginners
in programming, there was the concern to create a text capable of guiding the thoughts of the reader and make he to has the ability to generalize each technique for other problems. Among all optimizations possibilities, the most common in programming contests were chosen. Are they: Space reduction, RMQ data structures, Divide and Conquer Optimization, Knuth Optimization, Convex Hull Trick.
\end{abstract}

    % sumario
    \sumario % geracao automatica do sumario
    \chapter{Introdu\c{c}\~ao}
\label{chap:introducao}

Motivação:
- Uma das áreas mais importantes da maratona de programação
- Está em alta nas últimas competições


Justificativa:
- Poucos trabalhos  na área
- Livros não abordam esse assunto


Objetivos
- Gerar um trabalho que explique as principais técnicas de otimização utilizadas


\textcolor{red}{Alterações necessárias no modelo \LaTeX}

\begin{itemize}
	\item Adicionar uma lista de equações
	\item Tentar adicionar links no sumário
	\item Quadro != Tabela
\end{itemize}
    
\chapter{Fundamentação Teórica}
\label{chap:fundamentacao}

Neste capítulo, são apresentados os conceitos que servem de insumo para a elaboração das etapas seguintes desse trabalho. Na seção \ref{sec:complexidade} são demostrados alguns conceitos necessários para identificar e classificar a complexidade de um algoritmo. A seção \ref{sec:pd} explica o básico sobre programação dinâmica, seus principais termos e o conceito de otimização, que servirá de base para a elaboração dos capítulos seguintes. Por fim, no capítulo \ref{sec:ensino} é feita uma análise de alguns trabalhos relacionados com o ensino de programação.



\section{Complexidade de Algoritmos}
\label{sec:complexidade}
Na ciência da computação, analisar um algoritmo está relacionado com a identificação da quantidade de recursos necessários para sua execução, podendo ser a quantidade de memória utilizada, largura de banda de comunicação, hardware do computador. Porém mais frequentemente a preocupação maior é em se medir o tempo computacional gasto para realizar determinado código \cite{Cormen09a}.

Quando é feita a análise de complexidade, é possível identificar qual classe um determinado algoritmo pertence. A tabela \ref{tab:classes} lista algumas classes de forma ordenada, sendo a primeira função a melhor possível e a última a pior. Após classificar um algoritmo, é possível escolher entre diversas soluções para um mesmo problema, qual é a mais adequada no momento e poder saber antes de executar quanto tempo e memória o algoritmo irá gastar quando o tamanho da entrada for $n$, onde $n$ corresponde, geralmente a quantidade de elementos que devem processados. Na figura \ref{fig:complexity} é possível observar o comportamento de algumas funções na medida que a quantidade de elementos aumenta.


\begin{table}[H]
	\centering
	\caption[Principais classes de funções para analisar algoritmos]{Principais classes de funções para analisar algoritmos}
	\label{tab:classes}
	\begin{tabular}{c|c}
		\hline \SPACE
		\textbf{Notação} & \textbf{Exemplo de algoritmos} \\ \hline \SPACE
		$O(1)$ & Determinar se um número é par ou ímpar \\ \hline \SPACE
		$O(log n)$ & Busca binária \\ \hline \SPACE
		$O(\sqrt{n})$ & Determinar se um número é primo \\ \hline \SPACE
		$O(n)$ & Procurar um elemento em um \textit{array} não ordenado \\ \hline \SPACE
		$O(n * log n)$ & \textit{Merge sort}\protect\footnotemark \\ \hline \SPACE
		$O(n^2)$ & \textit{Bubble sort}\protect\footnotemark \\ \hline \SPACE
		$O(n^3)$ & \textit{Floyd-Warshall}\protect\footnotemark \\ \hline \SPACE
		$O(n^c)$ & Encontrar o maior emparelhamento em um grafo \\ \hline \SPACE
		$O(c^n)_{c > 1}$ & Resolver o problema do caixeiro viajante\protect\footnotemark com programação dinâmica \\ \hline \SPACE
		$O(n!)$ & Resolver o problema do caixeiro viajante com força bruta \\ \hline
	\end{tabular}

	\fonte{Pr\'oprio Autor.}
\end{table}
\addtocounter{footnote}{-3}
\footnotetext{Ver mais em: http://quiz.geeksforgeeks.org/merge-sort/}
\addtocounter{footnote}{1}
\footnotetext{Ver mais em: http://quiz.geeksforgeeks.org/bubble-sort/}
\addtocounter{footnote}{1}
\footnotetext{Ver mais em: http://www.geeksforgeeks.org/dynamic-programming-set-16-floyd-warshall-algorithm/}
\addtocounter{footnote}{1}
\footnotetext{Ver mais em: http://www.geeksforgeeks.org/travelling-salesman-problem-set-1/}

\begin{figure}[H]
	\centering
	\caption[Gráfico das principais classes de complexidade]{Gráfico das principais classes de complexidade}
	\includegraphics[width=0.7\textwidth]{complexity.png} % <- formatos PNG, JPG e PDF
	\fonte{PERRETT, 2010\nocite{Perrett2010}}
	\label{fig:complexity}
\end{figure}

\section{Programação dinâmica}
\label{sec:pd}

Programação dinâmica é uma técnica que combina soluções de subproblemas, da mesma maneira que a divisão e conquista, que divide o problema em subproblemas, resolve cada um recursivamente e depois é feita a junção das soluções para resolver o problema original. Porém este método é normalmente utilizado quando os subproblemas se sobrepõem, ou seja, um mesmo estado é encontrado diversas vezes na etapa de divisão. Portanto, se fosse aplicado um algoritmo ingênuo de divisão e conquista, um mesmo estado seria resolvido várias vezes, aumentando o custo computacional do algoritmo \cite{Cormen09a}. 

Para resolver o problema de sobreposição, a técnica de programação dinâmica salva a resposta de todos os estados que vão sendo encontrados. Assim, no momento que se deparar com algo que já foi resolvido ela simplesmente retorna o valor que já estava armazenado. 

A sequência de Fibonacci é um exemplo de fácil entendimento de quando é necessário a utilização de programação dinâmica. Esta é uma sequência de números inteiros que tem seu início com 0 e 1, os termos subsequentes são uma soma dos dois últimos números. A sequência recebeu o nome do matemático Leonardo de Pisa, mais conhecido como Fibonacci, que no ano de 1202 descreveu o crescimento da população de coelhos utilizando esta sequência \cite{LiveScience2013}. Os primeiros termos são:
\begin{equation}
0, 1, 1, 2, 3, 5, 8, 13, 21, 34, 55, 89, 144, 233, 377, 610, ...
\label{eq:fib}
\end{equation}
podendo ser representada através da seguinte recorrência, onde $fib(i)$ representa $i$-ésimo termo da sequência:
\begin{equation}
fib(i)=
\begin{cases}
i &\text{se } i \leq{1},\\
fib(i - 1) + fib(i - 2) &\text{se } i > {1}.
\end{cases}
\label{eq:fibrecorrence}
\end{equation}

\begin{figure}[H]
	\centering
	\caption[Árvore de recursão do Fibonacci de 5]{Árvore de recursão do Fibonacci de 5}
	\includegraphics[width=0.7\textwidth]{fib5.png} % <- formatos PNG, JPG e PDF
	\fonte{SCHWARTZ, 2011\nocite{Schwartz2011}}
	\label{fig:fib5}
\end{figure}


Na figura \ref{fig:fib5} é apresentada a árvore de recursão gerada ao utilizar a equação \ref{eq:fibrecorrence} para o cálculo do $fib(5)$. Através dela, é fácil ver que diversos estados estão se repetindo, por exemplo: $fib(2)$ aparece três vezes e sempre que é encontrado ele é divido no $fib(1)$ e $fib(0)$, assim deixando a  complexidade deste algoritmo em $O(2^{N})$. Ao aplicar programação dinâmica neste algoritmo é possível reduzir a complexidade para $O(N)$, pois cada estado será expandido uma única vez.

O código a seguir mostra como seria a implementação da função sem a utilização de programação dinâmica.
\begin{lstlisting}[caption={Implementação Fibonacci sem programação dinâmica},label={lst:fibsimples}]
int fib(int i){
	if(i <= 1)
		return i;
	return fib(i - 1) + fib(i - 2);
}

\end{lstlisting}

Para otimizar o código e utilizar programação dinâmica basta incluir uma tabela que salva todos os estados. Sua inclusão faz uma alteração mínima no código, como é mostrado no algoritmo \ref{lst:fibpd}.


\begin{lstlisting}[caption={Implementação Fibonacci com programação dinâmica},label={lst:fibpd}]
#define MAX 20 
int tabela[MAX + 1]; 
					 					 
int fib(int i){
	if(tabela[i] != -1) 
		return tabela[i];
	if(i <= 1)
		return tabela[i] = i;
	return tabela[i] = fib(i - 1) + fib(i - 2);
}
\end{lstlisting}

Para mais informações sobre programação dinâmica e suas técnicas, o site \textit{TopCoder}\footnote{https://www.topcoder.com/community/data-science/data-science-tutorials/} possui um artigo amplo com vários problemas e dicas para soluções. Ele divide sua explicação em teoria e prática, começando nos tópicos mais simples e indo até alguns mais avançados.

\subsection{Otimizações}

Ao utilizar programação dinâmica para otimizar um problema, normalmente ocorre uma queda drástica na classe de complexidade associada a solução, como é o caso da sequência de Fibonacci, discutida na seção \ref{sec:pd}, onde foi possível sair de uma complexidade exponencial para uma linear. Apesar de parecer uma ótima forma de solucionar um problema, às vezes apenas aplicar programação dinâmica não é suficiente, e existem casos onde é possível e necessário otimizar ainda mais.

Para utilizar uma técnica de otimização de programação dinâmica, alguns critérios em relação a função de recorrência devem ser correspondidos. Cada técnica tem uma abordagem que possibilita a resolução de um conjunto de problemas que compartilham certas propriedades.

A tabela \ref{tab:otimizacoes} ilustra algumas formas otimizações existentes, que possibilitam tanto na redução de espaço, quanto de tempo. Estas técnicas serão discutidas no capítulo \ref{chap:desenvolvimento}.

\begin{table}[H]
	\centering
	\caption[Otimizações de programação dinâmica]{Otimizações de programação dinâmica}
	\label{tab:otimizacoes}
	\begin{tabular}{p{4cm} | p{11cm}}
		\hline \SPACE
		\textbf{Nome/Tipo} & \textbf{Característica} \\ \hline \SPACE
		Redução de espaço &  Reduz a quantidade de memória necessária quando um estado depende de uma quantidade fixa de outros estados \\ \hline \SPACE
		Estrutura de dados &  Reduz a complexidade de tempo com o auxílio de uma estrutura de dados que consegue responder a consultas do tipo mínimo ou máximo em um intervalo de uma \textit{array}  \\ \hline \SPACE
		\textit{Divide and Conquer} & Realiza divisão e conquista para encontrar o ponto ótimo necessário para se resolver o estado atual, reduzindo a complexidade temporal \\ \hline \SPACE
		\textit{Knuth Optimization} & Utiliza informações de onde estava a solução ótima de um estado anterior para diminuir o espaço de busca dos outros, assim reduzindo a complexidade temporal  \\ \hline \SPACE
		\textit{Convex Hull Trick} & Através de conceitos geométricos, essa técnica possibilita a redução da complexidade temporal \\ \hline 
	\end{tabular}
	
	\fonte{Pr\'oprio Autor.}
\end{table}

\section{Ensino de algoritmos}
\label{sec:ensino}

O ensino de algoritmos, por se tratar de assuntos que normalmente não são tão simples, com um fundamentação matemática extensa, não é uma tarefa fácil, portanto foi feita uma busca na literatura de trabalhos que têm por objetivo criar uma metodologia de ensino. Assim servindo de modelo para o que está aqui sendo desenvolvido.


\nocite{methods}
Szlávi e Zsakó (2003) apresentaram diversas metodologias relacionadas ao ensino de programação. O autor discuti sobre alguns modelos e exemplifica quando e para qual nível de estudante cada método será mais proveitoso ao ser aplicado. Estes métodos determinam a forma de estruturar o curso que deseja ser ensinado e a maneira de explicar os conteúdos. Os métodos discutidos no trabalho são:

\begin{itemize}
\itemsep0em
\item Metodológico, orientado a algoritmos: Neste há uma divisão bem clara em todas as etapas do processo de programação, desde a estruturação do que será desenvolvido, até a documentação final. Este é recomendado para ser utilizado com estudantes no final do ensino médio ou que estejam se preparando para um trabalho na área de informática.
\item Orientado a dados: Este método é similar ao anterior, porém aqui tem o foco nas estruturas de dados, assim, muitas vezes conseguindo evitar os grandes teoremas que normalmente são complexos.
\item Orientado a especificação: A especificação da forma que o programa deve funcionar é a parte mais significativa, todos os algoritmos são gerados sistematicamente através das rígidas instruções. Método indicado para estudantes universitários de informática, pois este será bem sucedido se os alunos possuírem um profundo conhecimento matemático.
\item Orientado a problemas: Este método possui grandes diferenças do outros, neste toda a etapa de programação é vista como uma única tarefa que não pode ser dividida. Dentre todos os métodos discutidos, este é o único recomendável para todos os tipos de estudantes, pois o foco é desenvolver nos alunos uma forma mais algorítmica de pensar sem estar preocupado com o treinamento profissional.
\item Orientado a linguagem: Método semelhante ao orientado a problemas, porém o ensino dependente totalmente de uma linguagem de programação, todos os ensinamentos são voltados especificamente para a linguagem adotada. Devido a estes fatores esse método está desatualizado e não é tão útil nos dias atuais.
\item Orientado a instruções: Semelhante ao orientado a linguagem, porém aqui é definido uma linguagem geral, uma forma de pseudocódigo.
\item Orientado a matemática: Os problemas a serem resolvido com esta técnica, são retirados da matemática, onde cada problema individual é relacionado com algum outro utilizando os princípios básicos da matemática. Esta não é uma metodologia eficaz para o ensino de programação, porém o ensino de matemática com o auxílio de programação pode ser útil devido a forma de raciocínio que as duas áreas possuem.

\end{itemize}

\nocite{doi:10.1076/csed.13.2.137.14200} 
Uma ampla pesquisa na literatura com o foco na parte educacional do estudo de programação foi feita. Diversos métodos e tópicos foram identificados e analisados para poder ser realizado uma classificação, e assim, auxiliar os professores a identificar em seus alunos características comuns e padrões, que poderão ser contornados com base no que já foi realizado e está documentado na literatura, é mostrado por Robins, Rountree (2003).

\nocite{Pears:2007:SLT:1345443.1345441}
Pears \textit{et al.} (2007) desenvolveu um \textit{survey} que reúne algumas formas da literatura de ensinar a introdução de programação. Além disso, os trabalhos reunidos, foram classificados e agrupados pela forma de ensino e pelos métodos aplicados.


\nocite{teachingapplications} 
Zsakó e Nóra (2008) realizaram uma análise nos principais métodos e aplicações que auxiliam no aprendizado e no ensino dos tópicos de\sigla{ICT}{Information and Communication Technology}(do inglês, \textit{Information and Communication Technology}). Para cada método é exemplificado o seu funcionamento, como realizar a sua aplicação e para qual nível de estudante ele é mais apropriado. 


\nocite{teachingapplicationslanguages} 
No trabalho desenvolvido por Papp-varga, Szlávi e Zsakó (2008), foi feita uma análise semelhante a que foi realizada no trabalho apresentado na subseção anterior. Porém, o foco deste é o ensino de uma linguagem de programação, portanto os métodos apresentados demostram os passos ideais para transmitir os conceitos da linguagem proposta.

\nocite{newapproach} 
Radošević, Orehovački e Lovrenčić (2009), se propõem em criar uma ferramenta que facilite o aprendizado de linguagens de programação básicas, como C++, ao invés da utilização de\sigla{IDE}{Integrated Development Environment}(do inglês, \textit{Integrated Development Environment}). A finalidade desta ferramenta é ajudar os estudantes em não cometer erros que são comuns a quem está iniciando. Além disso, oferecer uma forma mais simples do professor auxiliar seus alunos.

\nocite{Vihavainen:2011:EAM:1953163.1953196}
Vihavainen, Paksula e Luukkainen (2011), discutem como ensinar o básico de programação para quem está começando. O autor propõe um modelo de ensino e faz a aplicação deste em uma turma de um curso de ciência da computação. O modelo discutido tem três estágios, no primeiro
o professor oferece aos estudantes um modelo conceitual de um problema e como o mesmo foi resolvido. A todo momento de explicação o professor fala todos os seus pensamentos e explica o motivo de ter feito determinada decisão, deixando os alunos a acompanharem toda a sua linha de raciocínio. Na segunda etapa, o estudante é exposto a um problema que será resolvido com a orientação de um instrutor que não entrega diretamente a resposta, mas sim sugestões que façam os alunos serem capazes de descobrir as respostas para suas próprias perguntas. A última etapa é quando os alunos conseguem resolver as tarefas sozinhos e não precisam mais do auxilio de um instrutor. Ao final do trabalho é feito um comparativo entre uma turma que utilizou da metodologia proposta e uma que não usou, e os resultados foram positivos, mostrando que a quantidade de evasão e reprovação foram reduzidos.

Apesar de existirem diversos trabalhos que tratam sobre o ensino de algoritmos e programação, todos os encontrados têm um foco para um nível básico de conhecimento, levando em conta alunos que estão iniciando nesta área, portanto, não será aplicado nenhum deste diretamente. No capítulo \ref{chap:metodo} será discutido a metodologia elaborada para a construção do trabalho.
 




    
\chapter{Trabalhos relacionados}
\label{chap:historico}

Nesta seção são apresentados alguns trabalhos que tem um objetivo similar ao deste projeto. Em todos os encontrados a intenção do autor foi criar um conteúdo teórico sobre algoritmos e programação. Apesar de apenas um trabalho possuir conteúdo relacionado a programação dinâmica, todos eles são úteis para a elaboração da metologia a ser desenvolvida.

Tommasini (201-) em seu trabalho visa ensinar programação dinâmica para quem está iniciando nesta área. Sua maior motivação foi a falta de um bom material didático sobre esse tema. Seu trabalho tem um foco muito didático, apresentando diversas técnicas, problemas com soluções e propondo vários exercícios para o leitor praticar o que foi ensinado. Ao final do trabalho é apresentado uma lista de problemas de maratonas de programação sobre os assuntos desenvolvidos no texto.


No trabalho proposto por Dalalio (2013), foi realizado um estudo de algoritmos e estruturas de dados para a resolução de problemas relacionados a \textit{String Matching}\footnote{https://en.wikipedia.org/wiki/String\_searching\_algorithm}. Seu texto é bem didático, mostrando diversos problemas e as diversas formas de resolução, apresentando a complexidade e seu código. Ao final diversos problemas são propostos para que o leitor os resolva utilizando os conceitos elaborados no texto.


Couto (2016) desenvolveu um trabalho voltado ao ensino de algoritmos, seu foco principal são os algoritmos aplicados em sequências de caracteres. Seu texto ficou dividido entre diversos capítulos, onde cada um deles apresentava uma estrutura de dados diferente, assim podendo mostrar ao leitor quando é melhor utilizar uma em relação as outras.






    
    
\chapter{Metodologia}
\label{chap:metodo}
\textcolor{red}{Explicar a linguagem que será utilizada (tanto do texto corrido, quanto a linguagem de programação, pseudocódigo, etc)}


    \chapter{Desenvolvimento (Alterar Nome) }
\label{chap:desenvolvimento}

Aqui entra o desenvolvimento real do trabalho com as principais técnicas de otimização

\section{Redução de espaço}
Dentre as otimizações disponíveis, as que envolvem redução de memória são as mais simples de serem aplicadas, seu uso pode ser facilmente entendido no problema da mochila \footnote{http://www.geeksforgeeks.org/dynamic-programming-set-10-0-1-knapsack-problem/}. Esse problema deseja maximar o valor dos itens colocados em uma mochila, onde estes possuem um valor e um peso associado, enquanto a mochila possui uma capacidade máxima de peso. Além disso, nenhum item pode ser dividido.  

\begin{equation}
dp[i][j] = 
\begin{cases}
0 &\text{se } i = 0 \text{ ou } j = 0,\\
max(valor[i-1] + dp[i-1][j-peso[i-1]], dp[i-1][j]) &\text{se } peso[i-1] \leq{j},\\
dp[i-1][j] &\text{se } peso[i-1] > j
\end{cases}
\label{eq:knapsack}
\end{equation}

O problema da mochila pode ser resolvido através da relação de recorrência apresentada acima, onde $dp[i][j]$ representa o valor máximo que pode se conseguir ao colocar os $i$-ésimos primeiros itens em uma mochila de capacidade $j$. Os vetores $valor$ e $peso$, representam o valor e peso associado a cada um dos $n$ itens, respectivamente. A resposta para o problema estará em $dp[n][capacidade]$.

Analisando a complexidade da equação \ref{eq:knapsack} é fácil ver que será necessário $O(n*capacidade)$, tanto de memória, quanto de tempo. Porém, é notório que para solucionar a linha $i$ da matriz de programação dinâmica, só são necessárias as respostas que já foram calculadas na linha $i - 1$, portanto podemos trabalhar apenas com duas linhas consecutivas da matriz, sempre alternando entre linha par e ímpar. 

\begin{equation}
dp[i\&1][j] = 
\begin{cases}
0 &\text{se } i = 0 \text{ ou } j = 0,\\

max(valor[i-1] + dp[\text{\textasciitilde}i\&1][j-peso[i-1]], dp[\text{\textasciitilde}i\&1][j]) &\text{se } peso[i-1] \leq{j},\\
dp[\text{\textasciitilde}i\&1][j] &\text{se } peso[i-1] > j
\end{cases}
\label{eq:knapsackmemorialinear}
\end{equation}

A equação \ref{eq:knapsackmemorialinear} demostra como reduzir a memória. Os valores que serão utilizados nas linhas da DP serão sempre 0 ou 1, assim o total de memória necessária é de $2*capacidade$, deixando com uma complexidade de $O(capacidade)$. A resposta para o problema da mochila utilizando esta relação estará em $dp[n\&1][capacidade]$.

O seguinte código mostra a implementação do problema da mochila com memória linear.
\begin{lstlisting}[caption={Implementação Mochila},label={lst:lis}]

int mochila(){
	int valor[] = {60, 100, 120};
	int peso[]  = {10, 20, 30};
	int capacidade = 50, n = 3;
	
	int dp[2][capacidade + 1];
	for(int i = 0; i <= n; i++){
		for(int j = 0; j <= capacidade; j++){
			if(!i || !j)
				dp[i&1][j] = 0;
			else if(peso[i - 1] <= j)
				dp[i&1][j] = max(valor[i-1] + dp[~i&1][j-peso[i-1]],
				                 dp[~i&1][j]);
			else
				dp[i&1][j] = dp[~si&1][j];
		}
	}
	return dp[n&1][capacidade]; 
}
\end{lstlisting}

\section{Estrutura de dados RMQ}

Um dos problemas clássicos de programação dinâmica é o\sigla{LIS}{Longest Increasing Subsequence}(do inglês, \textit{Longest Increasing Subsequence}). Neste, o objetivo é encontrar a maior subsequência de um \textit{array} onde todos os elementos estão ordenados de forma crescente.

Imaginemos o seguinte \textit{array}:

\begin{center}
	${2, 5, 3, 7, 11, 8}$
\end{center}

Uma das solução para este conjunto é a subsequência ${2, 3, 7, 8}$, que possui tamanho quatro.

Para resolver este problema pode ser usado programação dinâmica com a seguinte recorrência:

\begin{equation}
dp[i] = 
\begin{cases}
1 &\text{se } i = 0 \\
max(dp[j] + 1)_{0 \leq j < i} &\text{se } i \neq 0 \text{ e } v[j] \leq v[i]

\end{cases}
\label{eq:lisQuadrado}
\end{equation}

A equação \ref{eq:lisQuadrado}, resolve o problema do LIS, com a complexidade $O(n^2)$ de tempo e $O(n)$ de memória, porém é notório que quando está sendo calculado a $dp[i]$, ou seja, a maior LIS que termina no $i$-ésimo índice, está sendo percorrido todos elementos do $array$ que estão a esquerda e que possuem um valor menor ou igual que o elemento atual. Em outras palavras, queremos o maior elemento da $dp$ que seu valor está no intervalo $[0..v[i]]$.

A partir destas observações podemos melhorar a complexidade dessa solução com o auxílio de uma estrutura de dados que consegue consultar máximo ou mínimo de intervalo de um $array$ de forma mais eficiente. Pode ser ela uma $Segment Tree$\footnote{http://www.geeksforgeeks.org/segment-tree-set-1-range-minimum-query/} ou $BIT$\footnote{http://www.geeksforgeeks.org/binary-indexed-tree-or-fenwick-tree-2/}, com isso é possível remover a busca linear no \textit{array} pelo maior elemento e trocar por uma busca logarítmica, deixando a solução final com $O(n*logn)$.

O código a seguir mostra uma maneira de implementar o LIS utilizando $BIT$.
\begin{lstlisting}[caption={Implementação LIS},label={lst:lis}]

#define MAXN 20
int v[] = {2, 5, 3, 7, 11, 8};
int bit[MAXN];
int dp[MAXN];

void update(int x, int v){
	for(; x < MAXN; x+=x&-x)
		bit[x] = max(bit[x], v);
}

int get(int x){
	int ans = 0;
	for(; x; x-=x&-x)
		ans = max(ans, bit[x]);
	return ans;
}

int lis(int n){
	int ans = 0;
	
	for(int i = 0; i < n; i++){
		dp[i] = 1+get(v[i]);
		update(v[i], dp[i]);
		ans = max(ans, dp[i]);
	}
	
	return ans;
}
\end{lstlisting}

\section{Divide and Conquer Optimization}
Problemas:

\begin{itemize}
	\item Kattis Branch Assignment WF 2016
\end{itemize}

\section{Knuth Optimization}
Problemas:
\begin{itemize}
	\item Codechef = CHEFAOR
	\item URI 2475
\end{itemize}

\section{Convex Hull Trick}
Problemas:
\begin{itemize}
	\item URI 2481
\end{itemize}


    \chapter{Considerações Finais}
\label{chap:conclusao}

Considerações finais do trabalho, com conclusão e trabalhos futuros

\section{Conclusões}

\section{Trabalhos Futuros}
    % Aqui começa a bibliografia da monografia
    \bibliographystyle{abnt-alf}
    \bibliography{abnt-uefs}
    % Apêndice e Anexos
    \apendice
    %\input{postextuais/apendice}
    \anexo
    %\input{postextuais/anexo}
    \autoria % geração automática do termo de aprovação
\end{document} 