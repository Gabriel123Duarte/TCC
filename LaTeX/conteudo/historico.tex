
\chapter{Histórico e trabalhos relacionados}
\label{chap:historico}

\textcolor{red}{Capítulo destino a fazer o levanto sobre programação dinâmica, desde seu surgimento
até os dias atuais, mostrando os trabalhos na área e onde tem sido utilizada.}

Citar os trabalhos, dizendo o que cada um fez.

O "porém": nada focado nas otimizações X, Y e Z
- onde é usado além das maratonas


\section{Trabalhos Relacionados}

\textcolor{red}{Essa seção não deve existir na versão final}


\textbf{TCC Dalalio - String}: Trabalho realizado para o estudo de algoritmos e estruturas de dados para a resolução de problemas relacionados a "string matching". Seu texto é bem didático, mostrando diversos problemas e as diversas formas de resolução, apresentando a complexidade e seu código. Ao final diversos problemas são propostos para que o leitor os resolva utilizando os conceitos desenvolvidos no texto.

\textbf{TCC Stefano - DP}: O principal objetivo desse trabalho é o ensino de programação dinâmica para quem está começando nesta área. Sua maior motivação foi a falta de um bom material didático sobre esse tema. Seu trabalho tem um foco muito didático, apresentando diversas técnicas, problemas, soluções deles e propondo alguns exercícios para o leitor treinar o que foi ensinado. Ao final uma lista de problemas de maratonas são selecionados e categorizados para que o leitor resolva.

\textbf{TCC Yan - String}:  Trabalho voltado ao ensino de algoritmos, seu foco principal são os algoritmos aplicados em sequências de caracteres. Mostrando Autômatos, Tries, Árvores de sufixos. A ideia foi mostrar quando cada estrutura dessa pode ser aplicada.






